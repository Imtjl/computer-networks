\begin{table}[H]
	\centering
	\scalebox{0.75}{
		\renewcommand{\arraystretch}{1.2}
		\begin{tabular}{|>{\centering\arraybackslash}m{3.8cm}|
			>{\centering\arraybackslash}m{4cm}|
			>{\centering\arraybackslash}m{3cm}|
			>{\centering\arraybackslash}m{3cm}|
			>{\centering\arraybackslash}m{3cm}|
			>{\centering\arraybackslash}m{3.5cm}|}
			\hline
			\textbf{Метод кодирования}                                                         & \textbf{Спектральные характеристики} & \textbf{Постоянная составляющая} & \textbf{Самосин- хронизация} & \textbf{Обнаруж. ошибок} & \textbf{Сложность реализации} \\
			\hline
			\textbf{Потенциальный код с инверсией при единице (NRZI)}                          &
			Малая ширина спектра ($0$--$\frac{C}{2}$), итоговый спектр $\frac{3C}{7}$          &
			Присутствует, но снижена по сравнению с NRZ                                        &
			Отсутствует, возможны проблемы при длинных последовательностях '0'                 &
			Отсутствует                                                                        &
			Низкая сложность реализации, два уровня потенциала $\implies$ низкая стоимость                                                                                                                                                                         \\
			\hline
			\textbf{Манчестерское кодирование (М2)}                                            &
			Большая ширина спектра ($0$--$C$), итоговый спектр $C$                             &
			Отсутствует                                                                        &
			Отличная, переход в середине каждого битового интервала обеспечивает синхронизацию &
			Отсутствует                                                                        &
			Средняя сложность реализации, два уровня потенциала $\implies$ умеренная стоимость                                                                                                                                                                     \\
			\hline
			\textbf{Дифф. манчестерское кодирование}                                           &
			Большая ширина спектра ($0$--$C$), итоговый спектр $C$                             &
			Отсутствует                                                                        &
			Отличная, устойчива к инверсии сигнала                                             &
			Отсутствует                                                                        &
			Высокая сложность реализации, два уровня потенциала $\implies$ умеренная стоимость                                                                                                                                                                     \\
			\hline
			\textbf{NRZI с 4B/5B кодированием}                                                 &
			Уменьшенный итоговый спектр: $\frac{C}{3}$                                         &
			Отсутствует (благодаря 4B/5B)                                                      &
			Улучшена, длинные последовательности '0' полностью устранены                       &
			Отсутствует                                                                        &
			Низкая сложность реализации, простое сопоставление битов таблицей кодировки                                                                                                                                                                            \\
			\hline
			\textbf{NRZI со скремблированием}                                                  &
			Меньше спектр, чем у изначального NRZI, но больше 4B/5B: $\frac{2C}{5}$            &
			Отсутствует (благодаря скремблированию)                                            &
			Улучшена, благодаря скремблированию, но остались длинные последовательности '0'    &
			Отсутствует                                                                        &
			Средняя сложность реализации, считать полином для каждого преобразования долго и муторно                                                                                                                                                               \\
			\hline
		\end{tabular}
	}
	\caption{Сравнение методов кодирования из этапов 2, 3 и 4}
\end{table}

\textbf{NRZI} снижает постоянную составляющую по сравнению с NRZ и сохраняет простоту реализации. Однако его эффективность сильно зависит от статистики передаваемых данных; при длинных последовательностях '0' возможны проблемы с синхронизацией, а постоянная составляющая всё ещё присутствует. Тем не менее, по сравнению с AMI, который является его прямым конкурентом, но не был рассмотрен в данном отчёте, NRZI имеет лишь два уровня потенциала, что делает его дешевле и предпочтительнее для использования.

С помощью методов логического кодирования можно существенно улучшить NRZI.

\textbf{NRZI с 4B/5B кодированием} устраняет постоянную составляющую и улучшает синхронизацию благодаря введению избыточности. Длинные последовательности нулей заменяются кодами без длительных последовательностей нулей, что повышает надёжность передачи. Спектр после 4B/5B кодирования стал $\frac{C}{3} = 3.333 \, \text{МГц}$, что на 0.7 МГц меньше, чем при скремблировании того же метода кодирования, что существенно повышает скорость передачи сообщения.

\textbf{NRZI со скремблированием} лишь частично устраняет длинные последовательности одинаковых битов, улучшая синхронизацию и также устраняя постоянную составляющую. При этом сохраняется узкая полоса пропускания, а сложность реализации существенно выше, чем у избыточного кодирования 4B/5B, а также спектр выше на 0.7 МГц.

\textbf{Манчестерское кодирование} обеспечивает отличную синхронизацию и не имеет постоянной составляющей, что повышает надёжность передачи. Однако в сравнении с NRZI с 4B/5B кодированием, мы получаем на 2/3 C меньший спектр, при этом не проигрывая в синхронизации и постоянной составляющей. Таким образом, Манчестерское кодирование получается сложнее NRZI с 4B/5B кодированием, и дороже в реализации, не имея перед ним преимуществ.

\textbf{Дифференциальное манчестерское кодирование} наследует преимущества манчестерского кодирования и дополнительно устойчиво к инверсии сигнала, что важно в шумных средах. Хотя высокая сложность реализации и широкий спектр являются его недостатками, NRZI с 4B/5B кодированием не имеет устойчивости к инверсии сигнала - оно имеет лишь частичную возможность обнаружения ошибок за счёт наличия запрещённых символов. Исходя из этого, если важно иметь устойчивость к инверсии сигнала и есть возможность пожертвовать спектром, а соответственно более низкой скорости передачи сообщения, то предпочтительнее будет выбрать Дифференциальное манчестерское кодирование.


