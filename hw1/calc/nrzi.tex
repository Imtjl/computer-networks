Для определения верхней границы частот необходимо найти наиболее высокочастотную составляющую спектра в передаваемом сообщении, которая в NRZI образуется при передаче чередующихся значений 0 и 1, при этом период гармонического сигнала (синусоиды), используемого для передачи прямоугольных сигналов 0 и 1, будет равен удвоенной длительности битового интервала $\tau: T = 2\tau$, где $\tau$ определяется как величина, образная значению пропускной способности канала $C: \tau = \frac{1}{C}$. Отсюда верхняя граница частот будет равна \[f_{\text{в}} = \frac{1}{T} = \frac{C}{2}\]


В общем случае, при кодировании любого сообщения с помощью метода NRZI наибольшая (верхняя) частота достигается при передаче чередующихся значений 0 и 1, а наименьшая (нижняя) - при передаче длинных (в пределе - бесконечных) последовательностей нулей и единиц, что делает нижнюю границу частот близкой и в пределе равной нулю: $f_{\text{н}} = 0$. Следовательно, в предельном случае спектр: $S = f_{\text{в}} - f_{\text{н}} = f_{\text{в}} = \frac{C}{2}$.

С другой стороны, при передаче конкретного сообщения нижняя частота всегда больше нуля и зависит от максимальной длины последовательностей нулей. В этом случае для расчёта нижней границы чапстот необходимо в коде передаваемого сообщения найти \textit{наиболее длинную последовательность нулей}. В исходном сообщении, закодированном по методу NRZI, представленному на рисунке 1, низкочастотная составляющая образуется при передаче единицы и 6 последовательных нулей. Период синусоидального сигнала при передаче таких последовательностей равен 14 битовым интервалам и нижняя граница частот соответственно будет равна: $f_{\text{н}} = \frac{1}{14\tau} = \frac{C}{14}$. Тогда \textbf{спектр} при передаче данного сообщения кодом NRZI равен
\[
	S =  f_{\text{в}} - f_{\text{н}} = \frac{C}{2} - \frac{C}{14} = \frac{6C}{14} = 4.286 \, \text{МГц}
\]

То есть, при пропускной способности канала связи $C = 10 \, \text{Мбит/с}$ частота основной гармоники равна $f_{\text{в}} = \frac{10 \cdot 10^3}{2} = 5 \, \text{МГц}$, битовый интервал $\tau = 100 \, \text{нс}$, а нижняя частота гармоники $f_{\text{н}} = 0.714 \, \text{МГц}$.

Среднее значение частоты передаваемого сообщения находится в интервале $(f_{\text{н}};f_{\text{в}})$ и показывает, какие частоты (низкие или высокие) превалируют в спектре передаваемого сигнала.

Для оценки среднего значения частоты передаваемого сообщения можно для каждого битового интервала определить соответствующую частоту сигнала, просуммировать их и разделить на количество битовых интервалов. В нашем случае: частота основной гармоники $f_0 = \frac{C}{2}$ соответствует шести битовым интервалам, частота вдвое меньшая, т.е. $\frac{f_0}{2}$, соответствует одному битовому интервалу, частота $\frac{f_0}{3}$ - трём битовым интервалам, $\frac{f_0}{4}$ - двум битовым интервалам, $\frac{f_0}{6}$ - одному битовому интервалу и $\frac{f_0}{7}$ - также одному битовому интервалу.

Тогда средняя частота рассматриваемого сообщения
\[
	f_{\text{ср}} = \left(6f_0+\frac{f_0}{2}+3\frac{f_0}{3}+2\frac{f_0}{4}+\frac{f_0}{6}+\frac{f_0}{7}\right)/ 14 = \frac{349f_0}{504} = \frac{349 \cdot 5}{504} \approx 3.462 \, \text{МГц}
\]

Поскольку середине спектра рассматриваемого сообщения соответствует частота
\[
	f_{1/2} = (f_{\text{н}} + f_{\text{в}}) /2 = 2.857 \, \text{МГц}
\]
Можно констатировать, что в спектре сигнала \textit{незначительно превалируют высокие частоты}: $f_{\text{ср}} > f_{1/2}$.

Для качественной передачи двоичных сигналов по реальному каналу связи и возможности их распознавания на приёмной стороне с минимальным количеством ошибок, желательно на передающей стороне формировать сигналы, приближающиеся к прямоугольной форме. Однако, спектр таких сигналов оказывается слишком большим. Можно показать, что для качественного распознавания сигнала на приемной стороне при передаче чередующихся значений 0 и 1 достаточно сформировать сигнал, содержащий первые 4 гармоники (поскольку более высокочастотные гармоники оказывают незначительное влияние на результирующий сигнал) с частотами $f_0=\frac{C}{2}, f_1=3f_0, f_2=5f_0, f_3=7f_0$. В этом случае верхняя граница частот $f_{\text{в}}=7f_0$, а ширина спектра сигнала при передаче рассматриваемого сообщения соответственно будет равна $S = f_{\text{в}} - f_{\text{н}} = 7f_0-f_0/6=41f_0/6=34.167 \, \text{МГц}$. Полоса пропускания F, необходимая для качественной передачи данного сообщения, должна быть не меньше спектра S, например, $F = 35 \, \text{МГц}$.

Итак, при пропускной способности канала связи $C = 10 \, \text{Мбит/с}$ верхняя и нижняя границы частот в передаваемом сообщении равны соответственно $f_{\text{в}} = 5 \, \text{МГц}$ и $f_{\text{н}} = 0.714 \, \text{МГц}$, спектр сигнала $S = 4.286 \, \text{МГц}$, среднее значение частоты в спектре передаваемого сигнала $f_{\text{ср}} = 2.857 \, \text{МГц}$, полоса пропускания, необходимая для качественной передачи данного сообщения $F=35 \, \text{МГц}$.
