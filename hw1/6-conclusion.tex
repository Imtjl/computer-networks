В результате применения логического кодирования и скремблирования к исходному сообщению при использовании метода NRZI удалось значительно улучшить характеристики сигнала по сравнению с исходным методом NRZI. Благодаря избыточному кодированию 4B/5B и скремблированию были устранены длинные последовательности нулей, что повысило надёжность синхронизации и устранило постоянную составляющую сигнала. При этом сохраняется узкая полоса пропускания и низкая сложность реализации, что делает эти методы эффективными для использования в системах с ограниченными ресурсами и требованиями к полосе пропускания.

По сравнению с манчестерскими методами кодирования, NRZI с логическим кодированием или скремблированием обеспечивает более эффективное использование полосы пропускания, так как требует меньшей ширины спектра. Однако манчестерское и дифференциальное манчестерское кодирование всё ещё превосходят по надёжности синхронизации и устойчивости к помехам, что делает их предпочтительными в системах, где эти факторы являются критически важными.

Таким образом, применение логических кодирований и скремблирования позволяет улучшить характеристики методов с низкой сложностью реализации, не имеющих надёжной синхронизации и обнаружения ошибок, приближая их по качеству к более сложным методам, таким как манчестерское кодирование. Выбор между этими методами должен основываться на конкретных требованиях системы передачи данных, включая допустимую полосу пропускания, сложность реализации и необходимость в надёжной синхронизации.
